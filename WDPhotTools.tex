% rasti_template.tex 
%
% LaTeX template for creating an RASTI paper
%
% v1.0 released 12 November 2021
% (version numbers match those of rasti.cls)
%
% Copyright (C) Royal Astronomical Society 2021
% Authors:
% Peter Jones (OUP, adapted from mnras_template.tex, author Keith T. Smith (Royal Astronomical Society))

% Change log
%
% v1.0 November 2021
%    Adapted from mnras_template.tex


%%%%%%%%%%%%%%%%%%%%%%%%%%%%%%%%%%%%%%%%%%%%%%%%%%
% Basic setup. Most papers should leave these options alone.
\documentclass[fleqn,usenatbib]{rasti}

% RASTI is set in Times font. If you don't have this installed (most LaTeX
% installations will be fine) or prefer the old Computer Modern fonts, comment
% out the following line
\usepackage{newtxtext,newtxmath}
% Depending on your LaTeX fonts installation, you might get better results with one of these:
%\usepackage{mathptmx}
%\usepackage{txfonts}

% Use vector fonts, so it zooms properly in on-screen viewing software
% Don't change these lines unless you know what you are doing
\usepackage[T1]{fontenc}

% Allow "Thomas van Noord" and "Simon de Laguarde" and alike to be sorted by "N" and "L" etc. in the bibliography.
% Write the name in the bibliography as "\VAN{Noord}{Van}{van} Noord, Thomas"
\DeclareRobustCommand{\VAN}[3]{#2}
\let\VANthebibliography\thebibliography
\def\thebibliography{\DeclareRobustCommand{\VAN}[3]{##3}\VANthebibliography}


%%%%% AUTHORS - PLACE YOUR OWN PACKAGES HERE %%%%%

% Only include extra packages if you really need them. Common packages are:
\usepackage{graphicx}	% Including figure files
\usepackage{amsmath}	% Advanced maths commands
%% \usepackage{amssymb}	% Extra maths symbols
\usepackage{aas_macros}

%%%%%%%%%%%%%%%%%%%%%%%%%%%%%%%%%%%%%%%%%%%%%%%%%%

%%%%% AUTHORS - PLACE YOUR OWN COMMANDS HERE %%%%%

% Please keep new commands to a minimum, and use \newcommand not \def to avoid
% overwriting existing commands. Example:
%\newcommand{\pcm}{\,cm$^{-2}$}	% per cm-squared
\newcommand{\msun}{\mathcal{M}_{\sun}}

%%%%%%%%%%%%%%%%%%%%%%%%%%%%%%%%%%%%%%%%%%%%%%%%%%

%%%%%%%%%%%%%%%%%%% TITLE PAGE %%%%%%%%%%%%%%%%%%%

% Title of the paper, and the short title which is used in the headers.
% Keep the title short and informative.
\title[WD Photometric Toolkit]{WDPhotTools -- A White Dwarf Photometric Toolkit in Python}

% The list of authors, and the short list which is used in the headers.
% If you need two or more lines of authors, add an extra line using \newauthor
\author[M. C. Lam et al.]{
Marco C. Lam,$^{1}$\thanks{E-mail: lam@tau.ac.il}
A. N. Other,$^{2}$
and another Author$^{3}$
\\
% List of institutions
$^{1}$School of Physics and Astronomy, Tel Aviv University, Tel Aviv, Israel 69978
}

% These dates will be filled out by the publisher
\date{Accepted XXX. Received YYY; in original form ZZZ}

% Enter the current year, for the copyright statements etc.
\pubyear{2022}

% Don't change these lines
\begin{document}
\label{firstpage}
\pagerange{\pageref{firstpage}--\pageref{lastpage}}
\maketitle

% Abstract of the paper
\begin{abstract}
%%%%%%%%%%%%%%%%%%%%%%%%%%%%%%%%%%%%%%%%%%%%%%%%%%%%%%%%%%%%%%%%%%%%%%%%%%%%%%%%
From data collection to generating a white dwarf luminosity function, it
requires numerous Astrophysical, Mathematical and programming domain
knowledge. The steep learning curve makes it difficult to enter the field and
often individuals have to reinvent the wheel to perform identical data reduction
and analysis tasks. We have gathered all the publicly available white dwarf
cooling models and synthetic photometry to provide a toolkit that allows
visualisation of various models, photometric fitting of white dwarf with or
without parallax and reddening, and the computing of white dwarf luminosity
functions with a choice of initial mass function, star formation history,
initial-final mass relation, and white dwarf cooling models.

\end{abstract}

% Select between one and six entries from the list of approved keywords.
% Don't make up new ones.
\begin{keywords}
keyword1 -- keyword2 -- keyword3
\end{keywords}

%%%%%%%%%%%%%%%%%%%%%%%%%%%%%%%%%%%%%%%%%%%%%%%%%%
%%%%%%%%%%%%%%%%% BODY OF PAPER %%%%%%%%%%%%%%%%%%

\section{Introduction}
%%%%%%%%%%%%%%%%%%%%%%%%%%%%%%%%%%%%%%%%%%%%%%%%%%%%%%%%%%%%%%%%%%%%%%%%%%%%%%%%
White dwarfs~(WDs) are the final stage of stellar evolution of main
sequence~(MS) stars with zero age MS~(ZAMS) mass less than $8\msun$. Since this
mass range encompasses the vast majority of stars in the Galaxy, these
degenerate remnants are the most common final product of stellar evolution,
thus providing a good sample to study the history of stellar evolution and star
formation in the Galaxy. In this state, there is little nuclear burning to
replenish the energy they radiate away. As a consequence, the luminosity and
temperature decrease monotonically with time. The electron degenerate nature
means that a WD with a typical mass of $0.6\mathcal{M}_{\sun}$ has a similar
size to the Earth, giving rise to their high densities, low luminosities, and
large surface gravities.

%%%%%%%%%%%%%%%%%%%%%%%%%%%%%%%%%%%%%%%%%%%%%%%%%%%%%%%%%%%%%%%%%%%%%%%%%%%%%%%%
The use of the white dwarf luminosity function~(WDLF) as cosmochronometer was
first introduced by \citet{1959ApJ...129..243S}. Given a finite age of the
Galaxy, there is a minimum temperature below which no white dwarfs can reach in
a limited cooling time. This limit translates to an abrupt downturn in the WDLF
at faint magnitudes. Evidence of such behaviour was observed by
\citet{1979ApJ...233..226L}, however, it was not clear at the time whether it
was due to incompleteness in the observations or to some defect in the
theory~(e.g.,~\citealp{1984ApJ...282..615I}). A decade later,
\citet{1987ApJ...315L..77W} gathered concrete evidence for the downturn and
estimated the age\footnote{``Age'' refers to the total time since the oldest
WD progenitor arrived at the zero-age main sequence.} of the disc to be
$9.3 \pm 2.0$\,Gyr~(see also \citealt{1988ApJ...332..891L}). While most studies
focused on the Galactic discs~\citep{1989LNP...328...15L, 1992ApJ...386..539W,
1995LNP...443...24O, 1998ApJ...497..294L, 1999MNRAS.306..736K,
2012ApJS..199...29G}, some worked with the stellar
halo~\citep{2006AJ....131..571H, 2011MNRAS.417...93R, 2017AJ....153...10M,
2019MNRAS.482..715L}. 
 
%%%%%%%%%%%%%%%%%%%%%%%%%%%%%%%%%%%%%%%%%%%%%%%%%%%%%%%%%%%%%%%%%%%%%%%%%%%%%%%%
Most WDs have similar broadband colour to main sequence stars, they cannot be
identified using photometry alone. They were found from UV-excess, large
proper motion and/or parallax. Because of the strongly peaked surface gravity
distribution of white dwarf, photometric fitting for their intrinsic properties
is possible by assuming a surface gravity. WDs fitted in such way are useful
statistically provided that the sample is not strongly biased. This is
demonstrated in various studies comparing photometric and spectroscopic
solutions to calibrate atmosphere model, as well as from the agreeing shapes
of the WDLFs from spectroscopic and photometric samples. The Gaia satellite
provides parallactic measurement for over a billion point sources (references).
of which $359,000$ are high confidence WD candidates. The availability of
parallaxes allow much more accurate fitting, particularly without knowing
the surface gravity for the photometric sample. This has completely 
revolutionarized the field of WD sciences. In the forthcoming decade, the
Simonyi Survey Telescope at the Vera C. Rubin Observatory will continue to
discover more WDs at fainter magnitudes, but with only proper motion at best.
Furthermore, at those magnitudes, it is infeasible to collect spectrum for
most of them and thus studies will mostly rely on photometric methods.

%%%%%%%%%%%%%%%%%%%%%%%%%%%%%%%%%%%%%%%%%%%%%%%%%%%%%%%%%%%%%%%%%%%%%%%%%%%%%%%%
In this generation of user-side Astronomy data handling and analysis, as well
as the emphasis in computing courses for scientists, \texttt{Python} is among
the most popular languages due to its ease to use with a shallow learning
curve, readable syntax and simple way to "glue" different pieces of software
together. Its flexibility to serve as a scripting and an object-oriented
language makes it useful in many use cases: demonstrating with visual tools
with little overhead, prototyping, web-serving, and it can be compiled if
wanted. This broad range of functionality and high-level usage make it
relatively inefficient. However, Python is an excellent choice of language to
build wrappers over highly efficient and well-established codes. In fact,
some of the most used packages, scipy~\citep{2020NatMe..17..261V} and
numpy~\citep{2020Natur.585..357H}, are written in Fortran and C respectively.

%%%%%%%%%%%%%%%%%%%%%%%%%%%%%%%%%%%%%%%%%%%%%%%%%%%%%%%%%%%%%%%%%%%%%%%%%%%%%%%%
In this paper, it is organised as follow: in Section \textsection2, we explain
the reason behind this software and its structure. Section \textsection3 covers
the photometric fitting procedures and the overview of the model used. Section
\textsection4 explains the construction of a theoretical luminosity function
including some descriptions of the models available. We conclude this
work in Section \textsection5.

\section{Software Organisation}
%%%%%%%%%%%%%%%%%%%%%%%%%%%%%%%%%%%%%%%%%%%%%%%%%%%%%%%%%%%%%%%%%%%%%%%%%%%%%%%%
The goal for this work is to ease researchers the effort in reinventing the
wheel for trivial, repetitive but essential tasks in many aspect of data
analysis for WD science. This also allow simpler setup to compare results as a
result of the choice of models. There is already a spectroscopic version
for a similar purpose, the \texttt{wdtools}
\footnote{\url{https://github.com/vedantchandra/wdtools}}~
\citep{2020MNRAS.497.2688C}.

\textsc{WDPhotTools} can generate colour-colour diagram, colour-magnitude
diagram in various photometric systems, plotting cooling profiles from
different models, and compute theoretical white dwarf luminosity functions
based on the built-in or supplied models. The core parts of this work are
three-fold: (1)~the backbone of this work is the tailored-formatters that
handle the output models from various works in the format as they were
generated and downloaded. This allows the software to be updated with the
newer models easily in the future; (2)~photometric fitter that solves for
the WD parameters based on the photometry, with or without distance and
reddening; and (3) a class to generate white dwarf luminosity function in
bolometric magnitudes or in any of the photometric systems available from the
atmosphere model.

We host our source code on Github, which provides version control and other
utilities to facilitate the development. It uses \textsc{git}, issue and bug
tracking, high-level project management, automation with Github Actions upon
each commit for:

\begin{enumerate}
    \item Continuous Integration (CI) to install the software in Linux, Mac
    and Windows system, and then perform unit tests with \textsc{pytest}
    (Krekel et al. 2004)
    \item generating test coverage report with Coveralls which identifies
    lines in the script that are missed from the tests,
    \item Continuous Deployment (CD) through \textsc{PyPI} that allows
    immediate availability of the latest numbered version, and
    \item generating API documentation powered by \textsc{Sphinx} hosted on
    Read the Docs.
\end{enumerate}

% 
%@misc{pytestx.y,
%  title =        {pytest x.y},
%  author = {Krekel, Holger and Oliveira, Bruno and Pfannschmidt, Ronny and Bruynooghe, Floris and Laugher, Brianna and Bruhin, Florian},
%  year =         {2004},
%  url = {https://github.com/pytest-dev/pytest},
%}

\subsection{Visualisation Tools}
%%%%%%%%%%%%%%%%%%%%%%%%%%%%%%%%%%%%%%%%%%%%%%%%%%%%%%%%%%%%%%%%%%%%%%%%%%%%%%%%
Plotting tools are available to visualise various parameters of different
atmosphere and cooling models. Both the photometric fitting and WDLF generator
come with potting functions for diagnostics and final results. See the
respective sections for the example plots.

\subsection{Photometric Fitting}
Photometric fitting can be done by minimizing the $\chi^2$ with any method
supported by \verb+scipy.optimize.minimize()+, or with a Markov chain Monte
Carlo method powered by \texttt{emcee}~\citep{2013PASP..125..306F} with the
option to refine the solution with \verb+minimize()+ at the end using the
50-percentile as the initial guess and bounding the fit within a 1-sigma
uncertainty limit. The H and He atmosphere type has little effect on the
luminosity above around $6000$\,K. Below this, the assumption of an H
atmosphere for an He atmosphere star will lead to an overestimation in the
luminosity, and an underestimation in the distance. Optical spectra are
not particularly useful in distinguishing the atmosphere type, because below
$\sim$\,$5000$\,K, absorption lines starts to disappear. However, the
broad spectral features, namely the collisionally induced absorption, in the
infrared can distinguish the atmosphere type~(see Figure~12 of
\citealt{2017ApJ...848...36B}). Therefore, readers are reminded to pay
particular care when you arrive at a low temperature solution with only
optical data. Since it is only possible to fit for $n-1$ variables, if there
are only two photometric band, it is only possible to fit for a temperature
(and luminosity) with an assumed surface gravity and a known distance. With
three bands, it is mathematically possible to solve for an extra parameter,
i.e. assuming a surface gravity and fit for the temperature and distance,
or assuming a distance and fit for the temperature and surface gravity.
This argument does not apply to fitting with or without interstellar reddening
because it is merely a one-to-one lookup value that is a only a function of
distance. When studying a population of WDs that are fitted for the
distance, we should compare the photometric distance to the distances measured
from other means, e.g. astrometry, spectroscopic distance etc. in order to
assess the bias in the fit~(e.g. Section \textsection3.2.2
in \citealt{2011MNRAS.417...93R}).

\subsection{Theoretical WDLF}
WDLF is a common tool for deriving the age of a stellar population. It is the
number density of WD as a function of luminosity, which is an evolving function
with time, with its shape and normalisation determined from only a few
parameters. \citet{1987ApJ...315L..77W} compared an observed WDLF derived from
the Luyten Half-Second~(LHS) catalogue with a theoretical WDLF to obtain an
estimate of the age of the Galaxy for the first time with this technique.
\citet{1990ApJ...352..605N} examined WDLFs with various SFH scenarios. They
showed that WDLF is a sensitive probe of the star formation history~(SFH) as
it shows signatures of irregularities in the SFH such as bursts and lulls.
\citet{2013MNRAS.434.1549R} took it further to address this inverse problem
mathematically and showed some success in recovering the SFH of the solar
neighbourhood when compared against SFH computed from other methods.

The mathematical construction of a WDLF is intuitively straightforward: stars
were formed in a distribution of mass~($\mathcal{M}$), described by the initial
mass function~(IMF, $\phi$). Then, they spend their lifetime carrying out
nuclear burning~($t_{\mathrm{MS}}$), the time they spend depends mainly on
their mass. Towards the end stage of stellar evolution stars shed most of the
atmosphere, which is modelled by the initial-final mass relation~(IFMR,
$\zeta$). Once they have become WDs, all is left is to know how long it has
been cooling~($t_{\mathrm{cool}}$) in order to reach the current
luminosity~($\mathrm{M}_\mathrm{bol}$). The heavy duty of these computations
are coming from interpolation of pre-computed lookup tables. The important
part of this work is to carefully interpolate and integrate over the model
grids, because they are both susceptible to significant rounding errors given
the huge dynamic ranges the variables cover. For example, in case of a simple
star burst of $\mathcal{O}(10^6)$\,yrs, it requires a relative error
tolerance of $10^{-10}$ in order to integrate properly for an old population.

\section{photometric fit}
%%%%%%%%%%%%%%%%%%%%%%%%%%%%%%%%%%%%%%%%%%%%%%%%%%%%%%%%%%%%%%%%%%%%%%%%%%%%%%%%
The effective temperature (and absolute bolometric magnitude) and the solid
angle of a source can be obtained when the distance is known, through the
relation $\Omega = \pi R^2 / D^2$. In the era of
Gaia~\citep{2021A&A...649A...1G}, most WDs candidates have their parallaxes
measured, and hence the distances derived~\citep{2021AJ....161..147B}. This
allows much more reliable photometric fitting with broadband photometry alone,
as compared to fitting with an assumed surface gravity~(see below). By using
evolutionary model of WDs, the radius and the effective temperature can be used
to obtain the mass (and surface gravity). This concept is embedded into the
fitting procedure by interpolating the synthetic photometric grid models:
we provide two means of fitting, the first one is by a least-squared method
described by the equation

\begin{equation}
    \label{eq:lsq}
    \mathrm{minimize} \left[ \dfrac{\mathrm{mag}_i - \mu - \mathrm{model}_i}{\sigma_i} \right]^2
\end{equation}

where the index i denotes the filters, mag is the ovserved magnitude, $\mu$ is
the distance modulus, model is the syntehtic photometry in absolute magnitude,
and
\begin{equation}
    \sigma_i^2 = \sigma^2_{\mathrm{mag}, i} + \left[ \dfrac{5 \log(e)}{\varpi} \right]^2 \sigma_\varpi^2
\end{equation}
where $\sigma_{\mathrm{mag}, i}$ is uncertainty in the filters,
$\sigma_\varpi$ is that in parallax, and $\varpi$ is the parallax.

The second method we provide is to sample the parameter space with an MCMC
method. This method more useful in the case when distance is uncertain, or
lacking. The likelihood that has to be maximised takes a similar form to
Eq.~\ref{eq:lsq}:
\begin{equation}
    \label{eq:likelihood}
    \mathcal{L} = -\dfrac{1}{2} \sum_{i} \left\{ \left[ \dfrac{\mathrm{mag}_i - \mu - \mathrm{model}_i}{\sigma_i} \right]^2 + \ln(2\pi\sigma_i^2) \right\}.
\end{equation}


\subsection*{Synthetic Photometry}
%%%%%%%%%%%%%%%%%%%%%%%%%%%%%%%%%%%%%%%%%%%%%%%%%%%%%%%%%%%%%%%%%%%%%%%%%%%%%%%%
The only synthetic photometry publicly available over a smooth grid of wide
ranges of temperature and surface gravity is the Montreal
model\footnote{\url{https://www.astro.umontreal.ca/~bergeron/CoolingModels/}}.
The tables are provided with bolometric and absolute magnitudes on various
photometric systems in pure hydrogen~(DA) and pure-helium~(DB) in the range of
$\log(g)=7.0 - 9.0$, as well as the effective temperature and total cooling
time~(though this grid is sparse compared to their complementary cooling model
grids). These include the Johnson-Kron-Cousins~($U, B, V, R \& I$),
Two Micron All Sky Survey~(2MASS) $J, H \& K_{s}$, Mauna Kea Observatory~(MKO)
$Y, J, H \& K$, Wide-field Infrared Survey Explorer~(WISE) $W1, W2, W3 \& W4$,
Spitzer Space Telescope Infrared Array Camera~(IRAC)
$[3.6], [4.5], [5.8] \& [8.0] \mu m$, Sloan Digital Sky Survey~(SDSS)
$u, g, r, i \& z$, Panoramic Survey Telescope and Rapid Response
System~(Pan-STARRS 1) $g, r, i, z \& y$,
Gaia $G, G_{\mathrm{BP}} \& G_{\mathrm{RP}}$, and Galaxy Evolution
Explorer~(GALEX) $FUV$ and $NUV$.

The stellar masses and cooling ages are based on the latest generation of
evolutionary sequences~\citep{2020ApJ...901...93B}. The choice of
thick~($q_H \equiv \frac{M_H}{M*} = 10^{-4}$) and thin~($q_H = 10^{-10}$)
hydrogen layers for the pure-hydrogen and pure-helium model atmospheres,
respectively. Details of the colour calculations are described in
\citet{1995PASP..107.1047B} and \citet{2006AJ....132.1221H}. The DA grid covers
a temperature range $T_{\mathrm{eff}} = 2,500 - 150,000$\,K while the DB grid
covers $T_{\mathrm{eff}} = 3,250 - 150,000$\,K. Both model are computed with
surface gravities $\log(g) = 7.0 - 9.0$~\citep{2018ApJ...863..184B,
2020ApJ...901...93B, 2011ApJ...730..128T, 2011ApJ...737...28B, 2006ApJ...651L.137K}.

\subsection{Fitting Distance}
%%%%%%%%%%%%%%%%%%%%%%%%%%%%%%%%%%%%%%%%%%%%%%%%%%%%%%%%%%%%%%%%%%%%%%%%%%%%%%%%
Singly evolved WDs are distributed in a small range of surface gravity, with a
mean of $\left<\log(g)\right> = 7.998 \pm 0.011$ in the DA sample in SDSS
DR16~\citep{2021MNRAS.507.4646K}, corresponding to a mean mass of
$\left<M\right> = 0.618 \pm 0.006 \msun$. When studying a large sample of WDs,
fitting the photometric distances for the study of a population is still
useful when reporting an averaged quantity as the final results. The
distributions of the solution are, however, mostly statistical noise and not
representative to the true distribution when both the strongly degenerate
parameters: surface gravity and distance are fitted simultaneous. Equation
\ref{eq:lsq} and \ref{eq:likelihood} are reused in this case, however, the
distance modulus which is a function of distance, is a free parameter to be
fitted.

\subsection{Interstellar Reddening}
%%%%%%%%%%%%%%%%%%%%%%%%%%%%%%%%%%%%%%%%%%%%%%%%%%%%%%%%%%%%%%%%%%%%%%%%%%%%%%%%
When interstellar reddening is included in the calculation, the likelihood
function to be maximised becomes

\begin{equation}
    \mathcal{L} = -\dfrac{1}{2} \sum_{i} \left\{ \left[ \dfrac{\mathrm{mag}_i - \mu - \mathrm{model}_i - A_i(D)}{\sigma_i} \right]^2 + \ln(2\pi\sigma_i^2) \right\}
\end{equation}

where $A_i$ is the total extinction in filter $i$ at distance $D$. In the case
when the distance is known, this is a scale value; when the distance is also
to be fitted, $A$ is not directly provided but instead it is computed from the
$E(B-V)$ at that distance and a choice of $R_{V}$, which is defaulted to $3.1$.
The reddening vector is approximated by interpolating over the effective
wavelengths of the broadband filters\footnote{CTIO $UBVRI$, UKIRT $JHKL'$,
Gunn $griz$, SDSS $ugriz$, PS1 $grizy$, LSST $ugrizy$, DES $grizY$ and
WFC3 $F218W, F225W$ and $F275W$. We note that LSST is now renamed as
Simonyi Survey Telescope, at the Vera Rubin Observatory, but it was printed
under the former designation in the referenced article.} available in Table~6 of
\citet{2011ApJ...737..103S}. In this fitting, we strongly encourage the use of
MCMC sampling with a good number of walkers to sample the parameter space
properly as multiple local maxima of likelihood may exist.


%%%%%%%%%%%%%%%%%%%%%%%%%%%%%%%%%%%%%%%%%%%%%%%%%%%%%%%%%%%%%%%%%%%%%%%%%%%%%%%%
\subsection{Comparison against known work}


%%%%%%%%%%%%%%%%%%%%%%%%%%%%%%%%%%%%%%%%%%%%%%%%%%%%%%%%%%%%%%%%%%%%%%%%%%%%%%%%
\subsection{Joint Gaia EDR3--Pan-STARRS 1 reprocessing}


\section{White Dwarf Luminosity Function}
%%%%%%%%%%%%%%%%%%%%%%%%%%%%%%%%%%%%%%%%%%%%%%%%%%%%%%%%%%%%%%%%%%%%%%%%%%%%%%%%
The integral for a WDLF when parameterised with bolometric magnitude (as
opposed to luminosity) can be written as

\begin{equation}
    n(\mathrm{M}_{\mathrm{bol}}) = \int_{\mathcal{M}_l}^{\mathcal{M}_u} \tau(\mathrm{M}_\mathrm{bol}, m) \psi(T_0, \mathrm{M}_\mathrm{bol}, \mathcal{M}, m, Z) \phi(\mathcal{M}) d\mathcal{M}
\end{equation}
where $n$ is the number density, $\tau$ is the inverse cooling rate, $\psi$ is
the relative star formation rate, $\phi$ is the initial mass function; and their
dependent variables: $\mathrm{M}_\mathrm{bol}$ is the absolute bolometric
magnitude, $m$ is the WD mass, $T_0$ is the look-back time, $\mathcal{M}$ is
the progenitor MS mass, $Z$ is the metallicity, $\mathcal{M}_l$ is the minimum
progenitor MS mass that could have singly evolved into a WD in the given time,
and $\mathcal{M}_u$ is the maximum progenitor MS mass.

The inverse cooling rate
\begin{equation}
    \tau(\mathrm{M}_\mathrm{bol}, m) = \dfrac{dt_{\mathrm{cool}}}{d\mathrm{M}_\mathrm{bol}} \left( \mathrm{M}_\mathrm{bol}, m \right)
\end{equation}
is a quantity taken from the pre-computed grid of cooling models. 

The relative star formation rate is expressed as a function of look-back time,
\begin{equation}
    \psi(T_0, \mathrm{M}_\mathrm{bol}, \mathcal{M}, m, Z) = \psi\left[T_0 - t_{\mathrm{cool}}\left(\mathrm{M}_\mathrm{bol}, m\right) - t_{\mathrm{MS}}\left(\mathcal{M}, Z\right)\right].
\end{equation}
The absolute normalisation is not needed when the total stellar mass is coming
from observations; the theoretical WDLF only needs to multiply with a
constant (the total number density) to account for the normalisation.

The IFMR takes a simple form of
\begin{equation}
    m = \zeta(\mathcal{M}),
\end{equation}
although there are evidence that more metal rich stars lose more
envelope~\citep{2007ApJ...671..761K}, there is insufficient empirical data to
derive a IFMR at metallicity much lower or higher than solar abundance.

\subsection{Initial Mass Function}
%%%%%%%%%%%%%%%%%%%%%%%%%%%%%%%%%%%%%%%%%%%%%%%%%%%%%%%%%%%%%%%%%%%%%%%%%%%%%%%%
The IMF has little effect on the WDLF because of their similarity over the mass
range that stars could have become WDs. Nevertheless, we are providing three
of the most used IMFs, they differ slightly when the mass is less than $1\msun$.
The effect can appear in the bright end of a WDLF for an old population. A
callable function can be supplied as a manual IMF. This lists the four options
that are available:

\begin{enumerate}
    \item \citet{2001MNRAS.322..231K}
    \item \citet{2003PASP..115..763C}
    \item \citet[][including binary]{2003PASP..115..763C}
    \item Manual
\end{enumerate}

\subsection{Star Formation History}
%%%%%%%%%%%%%%%%%%%%%%%%%%%%%%%%%%%%%%%%%%%%%%%%%%%%%%%%%%%%%%%%%%%%%%%%%%%%%%%%
The SFH has one of the strongest effect on the shape and normalisation of a
WDLF. The default profiles are only three types of simple form SFH, however,
exponential decay profile with an appropriate decay coefficient should give a
first good guess for a disk population, while a burst profile would be useful
for studying open or globular clusters. All three profiles are controlled by
a few simple parameters, and the fourth option is to provide a callable
function of star formation rate as a function of look back time:
(1)~\textit{Constant} profile depends only on the age of the population,
i.e. the look back time since the beginning of star formation,
and \textbf{not} the time since the first WD was formed. (2)~\textit{Burst}
profile which depends on the onset of start formation as well as the duration
of a constant burst of star formation. (3)~An ~\textit{exponentially decaying}
profile that is government by the onset of star formation, the decay coefficient
and the duration of the star formation. By default it continues to decay
indefinitely. (4)~A ~\textit{manually provided callable function} that can take
any form, though users should be careful with the extrapolation setting,
smoothly extrapolated value or zero should be returned. In case of \verb+NaN+
or $\pm$inf being returned, we set the value to zero.

Since we normalise the output WDLF by the total integrated number density at
all luminosities, the absolute normalisation of the input SFH is discarded when
provided manually.

\subsection{Main Sequence Lifetime}
%%%%%%%%%%%%%%%%%%%%%%%%%%%%%%%%%%%%%%%%%%%%%%%%%%%%%%%%%%%%%%%%%%%%%%%%%%%%%%%%
The MS lifetime has a strong effect on the bright end of a WDLF because the hot
WDs spent most of their time since star formation as their progenitors.
However, the MS lifetime has decreasing impact on the WDLF as we move towards
the fainter end where WDs cooling time dominates over the MS lifetime. There is
also a metallicity dependency on the MS lifetime. From the PARSEC
models\citep{2013EPJWC..4303001B}, extremely metal poor solar mass star with
$Z=0.001$ and solar-like star with $Z=0.017$ take $\sim$\,$6$\,Gyr and
$\sim$\,$11$\,Gyr to go through the MS stage, respectively. Furthermore,
the mass loss during the late stage of stellar evolution also differ at
different metallicity, see the next part in this section. We provide the
following interpolated MS lifetime models and the option of providing a
callable function as a manual input:

\begin{enumerate}
    \item \citet{2013EPJWC..4303001B}
    \item \citet{2016ApJ...823..102C}
    \item Manual
\end{enumerate}

\subsection{Initial-Final Mass Relation}
%%%%%%%%%%%%%%%%%%%%%%%%%%%%%%%%%%%%%%%%%%%%%%%%%%%%%%%%%%%%%%%%%%%%%%%%%%%%%%%%
Human civilisation has existed a mere few thousand years, and modern astronomy
(with digital aid) is no more than a hundred. We do not have direct
observations on the total mass loss of a star at the end stage of stellar
evolution. It depends on observations of WDs and giants in clusters and
iteratively comparing with stellar evolution models. There are a number of
IFMRs available from studying globular clusters~\citep{2004A&A...420..515M,
2009ApJ...705..408K}, open clusters~\citep{2009ApJ...693..355W,
2016ApJ...818...84C}, wide MS-WD binaries~\citep{2008A&A...477..213C,
2012ApJ...746..144Z, 2018ApJ...860L..17E}, wide turnoff/subgiant-WD
binaries~\citep{2021ApJ...923..181B}, wide WD-WD
binaries~\citep{2015ASPC..493..325C, 2015ApJ...815...63A, 2018ApJ...866...21C}.
We have chosen six (plus three with two-part fitting) works that have a
good mass coverage to be included in this work, and just like the other
functions above, a manualy provided callable function is also accepted:

\begin{enumerate}
    \item \citet{2008MNRAS.387.1693C}
    \item \citet[][two-part]{2008MNRAS.387.1693C}
    \item \citet{2009ApJ...692.1013S}
    \item \citet[][two-part]{2009ApJ...692.1013S}
    \item \citet{2009ApJ...693..355W}
    \item \citet{2009ApJ...705..408K}
    \item \citet[][two-part]{2009ApJ...705..408K}
    \item \citet{2018ApJ...866...21C}
    \item \citet{2018ApJ...860L..17E}
    \item Manual
\end{enumerate}


\subsection{Cooling Models}
%%%%%%%%%%%%%%%%%%%%%%%%%%%%%%%%%%%%%%%%%%%%%%%%%%%%%%%%%%%%%%%%%%%%%%%%%%%%%%%%
Most of the internal energy of a WD is the residual heat from the progenitor
once it passed the planetary nebula phase. However, there are various physical
processes that can provide an appreciable amount of energy and govern the
cooling rate of a WD at different stages. Following the time sequence in which
the physical processes that has direct effects to the photo-luminosity: (1)~in
the first $10^8-10^9$ years, \textit{shell burning of hydrogen} via pp-chain
can contribute up to $30\%$ of the total luminosity~\citep{2010ApJ...717..183R}.
(2)~\textit{Neutrino losses} -- contribute to a significant fraction of energy
loss in the early time of WDs when they were still hot, in the case of massive
WDs, neutrino bremsstrahlung effect must also be taken into account
\citep{1994ApJ...425..222H, 1996ApJS..102..411I}. (3)~\textit{Gravitational
settling} of\ $^{22}$Ne in intermediate to massive WDs releases sufficient
gravitational potential energy to prolong the cooling
times~\citep{2002ApJ...580.1077D, 2008ApJ...677..473G, 2010ApJ...719..612A}.
The heavier\ $^{22}$Ne relative to the environment that is dominated by carbon,
oxygen and nitrogen leads to a slow settling towards the core. This effect is
the most obvious in the old and metal-rich systems, such as NGC
6791~\citep{2010Natur.465..194G, 2008ApJ...678.1279B}. (4)~In the late time of
the WD evolution, convection plays a significant role in slowing down the
cooling. As temperature decreases, the convective zone grows deeper into the
interior and eventually reaches the degenerate core~(see Figure~11
from~\citealt{2010A&ARv..18..471A}). This efficiently replenish the energy
radiated away from the photosphere, thus this process known as the
\textit{convective coupling}, modifies the relations between the WD luminosity
and core temperature~\citep{1989ApJ...347..934D, 2001PASP..113..409F}.
(5)~\textit{Crystallisation} occurs as the non-degenerate ions evolve from gas
to fluid and eventually solid. The liquid-solid transition releases latent
heat that slows down the cooling process. This also couples with the release
of gravitational energy associated with changes in the carbon-oxygen
profile~\citep{1997ApJ...486..413S} when the heavier oxygen-rich crystals
displace carbon as a result of gravitational settling. (6)~\textit{Coulomb
Interactions} modify the thermodynamical properties of the ionic gas, in
particular the specific heat. Its strength is determined by the Coulomb
coupling parameters. At first, the parameter is small, it slowly increases as
an WD cools and the ions begins to change for gas to liquid and eventually
forming lattice. This releases latent heat that contribute to $\sim$\,$5\%$
of the total luminosity~\citep{1976A&A....51..383S}. At late time, few modes
of the lattice are excited, the heat capacity drops according to the Debye law,
this results in enhanced cooling. This process kicks in after $10^9$\,yr for a
$1.0\,\msun$ WD and over a Hubble time for a $0.5\,\msun$ WD.

%%%%%%%%%%%%%%%%%%%%%%%%%%%%%%%%%%%%%%%%%%%%%%%%%%%%%%%%%%%%%%%%%%%%%%%%%%%%%%%%
We have included 11 cooling models that cover different parts of the parameter
space. See Table~\ref{tab:cooling_models} for the details of each model.

\subsection{Comparison of various choice of models}

\subsection{Reconstruction of WDLFs based on known SFH}


\begin{table*}
    \centering
    \begin{tabular}{c|c|c|c|c|c|c|c}
        Reference             &    Low     & Intermediate &    High    &  Core & Atmosphere &           Mass Range $\left(\mathcal{M}/\msun\right)$ & Extra Notes \\\hline\hline

        \multicolumn{8}{c}{Montreal Model} \\\hline
        \citet{2020ApJ...901...93B} & \checkmark &  \checkmark  & \checkmark &    CO &      DA/DB &            $0.2-1.3$             & -- \\
        &&&&&&&\\

        \multicolumn{8}{c}{La Plata Models} \\\hline
        \citet{2007MNRAS.382..779P} & \checkmark &      --      &     --     & He/CO &         DA &          $0.187-0.448$           & -- \\
        \citet{2009ApJ...704.1605A} & \checkmark &      --      &     --     &    He &         DA &          $0.220-0.521$           & -- \\
        \citet{2010ApJ...717..183R} &     --     &  \checkmark  &     --     &    CO &         DA &          $0.505-0.934$           & $Z=0.001-0.01$ \\
        {\citet{2015A&A...576A...9A}} &     --     &  \checkmark  &     --     &    CO &         DA &          $0.506-0.826$           & $Z=0.0003-0.001$ \\
        {\citet{2017A&A...597A..67A}} & \checkmark &  \checkmark  &     --     &    CO &         DA &          $0.434-0.838$           & $Y=0.4$ \\
        \citet{2017ApJ...839...11C} &     --     &  \checkmark  &     --     &    CO &         DB &           $0.51-1.0$             & -- \\
        {\citet{2007A&A...465..249A}} &     --     &      --      & \checkmark &   ONe &         DA &           $1.06-1.28$            & -- \\
        {\citet{2019A&A...625A..87C}} &     --     &      --      & \checkmark &   ONe &      DA/DB &           $1.10-1.29$            & -- \\
        &&&&&&&\\

        \multicolumn{8}{c}{BaSTI Models} \\\hline
        \citet{2010ApJ...716.1241S} &     --     &  \checkmark  & \checkmark &    CO &      DA+DB &           $0.54-1.2$             & With and Without Phase Separate\\
        &&&&&&&\\

        \multicolumn{8}{c}{MESA Models} \\\hline
        \citet{2018MNRAS.480.1547L} &     --     &      --      & \checkmark & CO/Ne &      DA/DB &          $1.012-1.308$           & --

    \end{tabular}
    \caption{The checkmarks in the low ($\mathcal{M}/\msun < 0.5$), intermediate
    ($0.5 < \mathcal{M}/\msun < 1.0$) and high mass ($1.0 < \mathcal{M}/\msun$)
    ranges denotes if the models are used for computation in that range.}
    \label{tab:cooling_models}
\end{table*}




\section{Conclusions}
%%%%%%%%%%%%%%%%%%%%%%%%%%%%%%%%%%%%%%%%%%%%%%%%%%%%%%%%%%%%%%%%%%%%%%%%%%%%%%%%


\section*{Acknowledgements}
%%%%%%%%%%%%%%%%%%%%%%%%%%%%%%%%%%%%%%%%%%%%%%%%%%%%%%%%%%%%%%%%%%%%%%%%%%%%%%%%


\section*{Data Availability}
%%%%%%%%%%%%%%%%%%%%%%%%%%%%%%%%%%%%%%%%%%%%%%%%%%%%%%%%%%%%%%%%%%%%%%%%%%%%%%%%


%%%%%%%%%%%%%%%%%%%% REFERENCES %%%%%%%%%%%%%%%%%%

% The best way to enter references is to use BibTeX:

\bibliographystyle{rasti}
\bibliography{WDPhotTools} % if your bibtex file is called example.bib


% Alternatively you could enter them by hand, like this:
% This method is tedious and prone to error if you have lots of references
%\begin{thebibliography}{99}
%\bibitem[\protect\citeauthoryear{Author}{2012}]{Author2012}
%Author A.~N., 2013, Journal of Improbable Astronomy, 1, 1
%\bibitem[\protect\citeauthoryear{Others}{2013}]{Others2013}
%Others S., 2012, Journal of Interesting Stuff, 17, 198
%\end{thebibliography}

%%%%%%%%%%%%%%%%%%%%%%%%%%%%%%%%%%%%%%%%%%%%%%%%%%

%%%%%%%%%%%%%%%%% APPENDICES %%%%%%%%%%%%%%%%%%%%%

\appendix

\section{Some extra material}

%%%%%%%%%%%%%%%%%%%%%%%%%%%%%%%%%%%%%%%%%%%%%%%%%%


% Don't change these lines
\bsp	% typesetting comment
\label{lastpage}
\end{document}

% End of rasti_template.tex
